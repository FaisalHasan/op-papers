\documentclass{article}
\usepackage{graphicx}
\usepackage{wrapfig}
\usepackage{inconsolata}
\usepackage{enumerate}
\usepackage{hyperref}
\usepackage[margin = 2.25cm]{geometry}




\begin{document}

\begin{figure}
\includegraphics[width=30mm]{../../../resources/images/oplogo.png}
\end{figure}

\title{Course Directive\\IN719 Systems Administration \\Semester One, 2019}
\date{}
\maketitle

\section*{Description}
In his opinion document for the 1964 Jacobells v. Ohio case in the United States Supreme Court, Justice Potter Stewart wrote:
\begin{quote}
I shall not today attempt further to define the kinds of material I understand to be embraced within that shorthand description; and perhaps I could never succeed in intelligibly doing so. But I know it when I see it, ...
\end{quote}
It turns out that Stewart was referring to hard-core pornography, but he could have easily been talking about systems administration. It's difficult to define precisely what systems administration is, but it is an important area of work in any organisation that relies on functional ICT infrastructure. In this paper we attempt to identify a solid core of modern systems administration practice and model it.



\section*{Course Information}
\begin{itemize}
  \item 15 Credits
  \item IN617 or equivalent
\end{itemize}

\section*{Lecturer}
\begin{tabular}{lr}

  % after \\: \hline or \cline{col1-col2} \cline{col3-col4} ...
  Tom Clark &    \\
     Email: & \texttt{tom@runawayplay.com, tom.clark@op.ac.nz} \\
     GitHub: & \url{https://github.com/tclark} 
\end{tabular}

\section*{Course Dates}
\begin{tabular}{ll}
Term 1 (8 weeks) & 18 February - 12 April\\
Term 2 (8 weeks) & 29 May - 28 June\\
\end{tabular}

\newpage 

\section*{Learning Outcomes}
On completion of this paper you will be able to:
\begin{enumerate}
  \item work as a member of a systems administration team following industry standard work practices;
  \item use a centralised configuration management system;
  \item monitor and track system states of multiple servers using a monitoring tool;
  \item perform packup and recovery tasks using a backup management system.
\end{enumerate}

\section*{Resources}
\begin{itemize}
	\item Course notes, lecture slides, and lab documents are availble in a GitHub repository published at \\ \url{https://github.com/tclark/op-papers}.
	\item We will access our systems remotely using ssh and Remote Desktop.
	\item We will use a ticketing system for work ticketing. We will set up your account during the first week.
	\item We will use MediaWiki for team and course documentation. We will set up your account during the first or second week.
	\item You will need a (free) GitHub account to manage your files.
	\item A Slack workspace has been created for this class at \url{https://in719.slack.com}. Invitations to join will be sent to your student email addresses.
\end{itemize}

\pagebreak

\section*{Course Content and Schedule}
This schedule is subject to change based on the needs of the class. A good sysadmin is flexible and responds to changes gracefully.

\renewcommand{\arraystretch}{1.5}
\begin{tabular}{|l|c|l|l|}
\hline
 Week & Week Start & Topic                              & Assessment         \\ \hline
 1    & 18 Feb     & Introduction, Ticketing            &                    \\ \hline
 2    & 25 Feb     & Documentation                      &                    \\ \hline
 3    &  4 Mar     & Linux Packaging, Active Directory  &                    \\ \hline
 4    &  11 Mar    & Configuration Management           &                    \\ \hline
 5    & 18 Mar     & Configuration Management           &                    \\ \hline
 6    & 25 Mar     & Configuration Management           &                    \\ \hline
 7    & 1 Apr      & System Monitoring                  &                    \\ \hline
 8    & 8 Apr     & System Monitoring                  & Performance Review \\ \hline
 H1   & 15 Apr     & Holiday                            &                    \\ \hline
 H2   & 22 Apr     & Holiday                            &                    \\ \hline
 9    &  29 Apr     & System Monitoring                  &                    \\ \hline
 10   &  6 May     & Backup and Recovery                &                    \\ \hline
 11   & 13 May     & Backup and Recovery                &                    \\ \hline
 12   & 20 May     & Lab Work                           &                    \\ \hline
 13   & 27 May     & Team Operations Assessment         &  Ops Assessment    \\ \hline
 14   &  3 Jun     & Team Operations Assessment         &  Ops Assessment    \\ \hline
 15   & 10 Jun     & Postmortem                         &                    \\ \hline
 16   & 17 Jun     & Final Assessment                   & Final              \\ \hline
\end{tabular}



\section*{Assessment}
There are three assessments in this paper, weighted as follows:


\begin{tabular}{|l|c|c|}
\hline
Assessment & Due Date & Weighting \\ \hline
Individual Performance Review & Week 8 & 20\% \\ \hline
Team Operational Assessment & Week 13 and 14 & 60\% \\ \hline
\end{tabular}


\section*{Criteria for Passing}
You must earn an overall average mark of 50\% or better to pass this paper. There must be a genuine attempt at all assessments. There are no resits.

\section*{Course Requirements and Expectations}
\subsection*{Attendance}
\begin{itemize}
 \item Students are expected to attend all classes, both lectures and labs.
 \item If you miss a class you should get notes from another student.
 \item If you cannot attend for two or more consecutive sessions, contact the lecturer.
 \item You must turn up ready for assessments on the due date and at the correct time. No 
       extra time will be scheduled. If you do not turn up, you have failed the assessment.
\end{itemize}

\subsection*{Communication}
Important announcements and discussions about the course, assessments, and scheduling may take place during class sessions.  It is your responsibility to be informed about them.  If you cannot attend a class session, be sure to check with another student.

The class Slack workspace is an official communication channel. You must join the channel and monitor it regularly. It will frequently be the main source of important updates and announcements and is also a place to discuss the class and get help.

Your student email is another official communication channel. It is your responsibility to regularly check your student email for important course related material, including changes to class scheduling or assessment details. Not checking will not be accepted as an excuse.

You can manage your email at the Student Hub and download the instructions for forwarding your email at \url{http://www.op.ac.nz/students/student-hub/}

\subsection*{Polytechnic Closure}
In the event that the Polytechnic is closed or has a delayed opening because of snow or bad weather you should not attempt to attend class if it is unsafe to do so. It is possible that your instructor will not be able to attend either, so classes may not physically meet. However, this does not become a holiday. Rather, material will be available on GutHub covering the classes affected by the closure. You are responsible for any material presented in this manner. Information about closure will be posted on the Otago Polytechnic Facebook page \url{https://www.facebook.com/OtagoPoly}.

\subsection*{Group Work and Originality}
Students in the Bachelor of Information Technology degree are expected to hand in original work.  Students are encouraged to discuss
assignments with their fellow students.  However, all assignments are to be completed as individual works unless group work is explicitly involved.
Failure to submit your own unique work will be treated as plagiarism.

\subsection*{Referencing}
Appropriate referencing is required for all work.  Referencing standards will be specified by your instructor.

\subsection*{Plagiarism}
Plagiarism is submitting someone else's work as your own.  Plagiarism offences are taken seriously and an
assessment that has been plagiarised may be awarded a zero mark.  A definition of plagiarism is in the Student Handbook,
available online or at the school office.

\subsection*{Submission Requirements}
All assignments are to be submitted by the time, date, and method given when the assignment is issued.
Failure to meet all requirements may result in a penalty of up to 10\% per day (including
weekends).

\subsection*{Extensions}
Extensions are only available for unusual circumstances.  These must be applied for, and approved, prior to the submission deadline.

\subsection*{Impairment}
In case of sickness contact your lecturer or year co-ordinator as soon as possible, preferably before the test or
assignment is due.  The policy regarding the granting of a mark that considers impaired performance requires a medical
certificate and a medical practitioners signature on a form. You may should refer to the guide on impaired performance
on the student handbook.

\subsection*{Appeals}
If you are concerned about any aspect of your assessment, please approach the lecturer in the first instance.  We support
an open door policy and aim to resolve issues promptly.  Further support is available from the Programme
Manager and Head of School. Otago Polytechnic has a formal process for academic appeals if necessary.

\subsection*{Other Documents}
Regulatory documents relating this course can be found on the Polytechnic website.

\subsection*{Special Resources and Requirements}
If you have any special needs, whether they relate to the course material, the exercises, the assessment, or anything in the course -
then \textit{please} let your instructor know as soon as possible.

\end{document}
