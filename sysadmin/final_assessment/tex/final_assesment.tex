\documentclass{article}   	% use "amsart" instead of "article" for AMSLaTeX format
\usepackage[margin=0.5in]{geometry}                		% See geometry.pdf to learn the layout options. There are lots.
\geometry{a4paper}                   		% ... or a4paper or a5paper or ...

\usepackage[parfill]{parskip}    		% Activate to begin paragraphs with an empty line rather than an indent
\usepackage{graphicx}				% Use pdf, png, jpg, or eps with pdflatex; use eps in DVI mode
\usepackage{enumerate}								% TeX will automatically convert eps --> pdf in pdflatex		


\title{Final Assessment\\ IN 719  Systems Administration}
\date{}							% Activate to display a given date or no date

\begin{document}
\maketitle

\section{Introduction}
The truth is, nobody loves a good exam more than I do. However, I've never felt like an exam was the right way to end this paper. It would seem out of place compared to everything else we have done this semester. So instead of a typical exam, we will take part in a mock job application process. We will do the assessment in two parts. First, you will submit a CV and cover letter applying for the job advertisement below. After that you will take a short quiz to establish your basic technical competence. Finally, you will schedule an interview with one of the lecturers - whichever one did your mid-semester review.

\section{Instructions}
You will ``apply'' for the following job:

\begin{quotation}
  EvilGlobalMegaCorp (EGMC) is seeking a number of junior level systems administrators to join our team as we scale up operations in preparation for the global launch of \emph{Purr}, a mobile/online dating application for cats.  This role will see you using your technical expertise to perform system administration tasks to manage the day-to-day ICT services for the organisation and to participate in bringing a number of new systems online.You will be responsible for the installation, configuration, maintenance and upgrades of server based systems and applications.
  
  We are looking for people with the following skills and experience:
  \begin{itemize}
    \item Experience in systems administration and working in a complex systems environment
    \item Linux experience or general Linux knowledge required 
    \item Experience with networking concepts
    \item Solid organisational skills; time management, managing and prioritising tasks
    \item Excellent written and verbal communication skills
    \item Relevant tertiary and industry qualifications preferred
  \end{itemize}
\end{quotation}

Note that this job is a lightly edited version of a real posting I found on Seek.

\subsection{Cover Letter/CV}

Submit your cover letter and CV as two PDF documents by email to \texttt{tclark@op.ac.nz} by 17:00 on Friday, 14 June.  Between now and that time we will answer questions and give feedback on drafts of these documents.  

\subsection{Quiz and Interview}
On Friday, 14 June at 9:00 you will take a short quiz in D202 or D313. The quiz will take about 15 minutes and is not meant to be particularly hard. It is meant to check for a basic level of knowledge and is modelled after quizzes that are sometime used in applicant screening processes in industry. We will discuss your quiz at the interview.
  
Between Tuesday, 11 June and Friday, 14 June you will sign up for an interview time with one of the lecturers. The interview will be conducted during the week of 17 June. You will have an interview that will take approximately 30 minutes. For the interview you should be prepared to answer a variety of questions about any of the topics covered during the semester. In addition, you may be asked about systems administration practices and problem solving.

\newpage

\section{Submission and marking}

The CV and cover letter are worth 30 marks. You will be evaluated on correct formatting, language, spelling, and effective communication. If you are unsure of how to prepare a CV or write a good cover letter, ask me. I have both written and read many of these over my career. 

The interview questions are worth 70 marks and will be evaluated using the same criteria as above. Note that the questions are less about knowing facts and more about showing good judgement and reasoning. Your goal is to show that you can think about things ``like a sysadmin''.

This assessment is worth 20\% of your overall mark for this paper.
 





\end{document}
