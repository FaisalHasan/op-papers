\documentclass{article}   	% use "amsart" instead of "article" for AMSLaTeX format
\usepackage[margin=0.5in]{geometry}                		% See geometry.pdf to learn the layout options. There are lots.
\geometry{a4paper}                   		% ... or a4paper or a5paper or ...

\usepackage[parfill]{parskip}    		% Activate to begin paragraphs with an empty line rather than an indent
\usepackage{graphicx}				% Use pdf, png, jpg, or eps with pdflatex; use eps in DVI mode
\usepackage{enumerate}								% TeX will automatically convert eps --> pdf in pdflatex		


\title{Operational Review\\ IN719 Systems Administration}
\date{}							% Activate to display a given date or no date

\begin{document}
\maketitle


On Monday, 27 May at 9:00 AM we will begin our approximately two week operations evaluation period which will end on Friday, 7 June at 17:00.  During this time we will not meet as a class.

You will be evaluated on the following criteria:

1.  \textbf{ownCloud Uptime (50 marks):}  Your primary goal is to keep ownCloud running.  If you keep ownCloud running for 260 hours during this period, you will receive 50 marks.  If you have less than 210 hours of uptime you will receive 0 marks.  Every hour between the two is worth 1 mark.  Uptime is measured by an instance of Nagios operated by the lecturers  Note that you will not lose marks if there is an outage in the VMWare platform or the Polytechnic network - but if you suspect that such a problem is occurring you should verify that this is the case.

If you do the arithmetic you'll see that there are two hours unaccounted for.  We will get to that.

2.  \textbf{Issues/Tickets (20 marks):}  During the evaluation period you will receive tickets to handle.  Some tickets will come from a standard user (user@foo.org.nz).  You are only expected to deal with those tickets during normal business hours (9-5 Monday-Friday).  You will also receive tickets from your manager (it-manager@foo.org.nz).  You are expected to handle those tickets between 8:00 and 22:00 any day of the week.  Note that Monday, 3 June is a holiday and so is not considered to include in normal business hours.

If you have any work commitments or other schedule constraints during the evaluation period, bring the to the lecturers' attentions before close of business on Friday, 24 May.

For these tickets you will be evaluated based on:
 
 \begin{itemize}
   \item effectively resolving the issues;
   \item handling them promptly;
   \item communicating with relevant parties
 \end{itemize}
 
3.  \textbf{Security Breach/System Failure (20 marks):}  At some point during the evaluation period you will have to deal with a major systems failure or a security breach.  (This is where the two hours of downtime come in).  You will be evaluated on:

 \begin{itemize}
  \item recognising the problem;
  \item solving the problem;
  \item communicating with affected parties (e.g., the two email addresses above).
\end{itemize}

4.  \textbf{Bacula/Nagios/Puppet setup (10 marks):} We will review your utilisation of Bacula, Nagios, and Puppet.  Note that we will do this near the end of the evaluation period, so you should use the time to improve your use of these tools if necessary.

During the evaluation (and before it starts)I will be available to answer questions and clarify things as needed.

You should post a schedule on your team wiki identifying a primary contact person from your team over the perios of the assessment. Each team member should take responsibility for half of the time.

\end{document}
