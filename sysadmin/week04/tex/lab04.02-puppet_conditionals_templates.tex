\documentclass{article}   	% use "amsart" instead of "article" for AMSLaTeX format
\usepackage[margin=0.5in]{geometry}                		% See geometry.pdf to learn the layout options. There are lots.
\geometry{a4paper}                   		% ... or a4paper or a5paper or ...

\usepackage[parfill]{parskip}    		% Activate to begin paragraphs with an empty line rather than an indent
\usepackage{graphicx}				% Use pdf, png, jpg, or eps with pdflatex; use eps in DVI mode
\usepackage{enumerate}
\usepackage{hyperref}								% TeX will automatically convert eps --> pdf in pdflatex		


\title{Lab 04.02: A Puppet Module for NTP \\ IN719 Systems Administration}
\date{}							% Activate to display a given date or no date

\begin{document}
\maketitle

\section*{Introduction}
Virtual machines may be the best thing since sliced bread. Just like sliced bread, however, they are very bad at keeping time accurately. This leads to some difficult problems when two or more machines need to coordinate the delivery of a service and they disagree on what time it is. Fortunately, this problem is easily solved by running \emph{Network Time Protocol} (NTP) services. NTP uses a hierarchical clinet-server achitecture. Top tier servers get their time from GPS or other satellites. Lower tier servers get their time from higher tier servers. Clients get their time from various NTP servers.

In this lab we will set up an ntp server on our respective \texttt{mgmt} servers and clients on the other servers. We only want to write one Puppet module, so we will use \emph{conditionals} and \emph{templates} in our module to accomplish this.

\section{Module setup}
Create a standard module structure in the \texttt{/etc/puppet/modules} directory of your puppetmaster.

\texttt{ntp} \\
\texttt{ntp/manifests} \\
\texttt{ntp/files} \\
\texttt{ntp/templates} \\

Create an \texttt{init.pp} file in your \texttt{manifests} subdirectory.

\section{Module manifest}
Put the following code in your \texttt{init.pp} file.  Adjust the \texttt{group00mgmt} hostname to match your own.

\begin{verbatim}

  class ntp_service {
      include install, config, service

  }

  class ntp_service::install {
      package { "ntp":
          ensure => present,
      }
  }

  class ntp_service::config {
      if $hostname == "group00mgmt" {
          $restrict = "restrict 10.25.0.0 mask 255.255.0.0 nomodify notrap"
          $server = "server 127.127.1.0"
          $fudge = "127.127.1.0 stratum 10"
      } else {
          $restrict = ""
          $server = "server group00mgmt.foo.org.nz prefer"
          $fudge = ""
      }
      file { "/etc/ntp.conf":
          ensure => present,
          owner => "root",
          group => "root",
          mode => 0444,
          content => template("ntp/ntp.conf.erb"),
      }    
  }

  class ntp_service::service {
      service { "ntp":
          ensure => running,
          enable => true,
      }    
  }


  
\end{verbatim}



There are a few new things happening in this manifest.
\begin{itemize}
  \item We're using\emph{variables}, such as \texttt{\$hostname}.  We can define and use our own variables, but many variables are populated for us by a utility called \emph{Facter}.  You can see a list of the core facts produced by Facter at \url{https://puppet.com/docs/puppet/5.3/lang_facts_and_builtin_vars.html}.
  \item We are using an \texttt{if/else} structure to conditionally populate variables based in the hostname of the agent.
  \item Instead of copying over static files, we are using \emph{templates}.  The template files are to be placed in the \texttt{templates} subdirectory of the module.
\end{itemize}

\section{Template files}
Retrieve the \texttt{ntp.conf.erb} file from the week 4 subdirectory of the class GitHub repository and copy it into the \texttt{templates} subdirectory of the ntp module. Examine the contents of the file and look for these lines:

\begin{verbatim}

  <%= restrict %>

  <%= server %>

  <%= fudge %>

\end{verbatim}

Each of these lines will be replaced by the values of the variables we set up in our manifest above. This template is actually just an \texttt{erb} template, the same kind that is typically used with Ruby on Rails. More documentation on using templates with Puppet is available at \url{https://puppet.com/docs/puppet/5.3/lang_template_erb.html}\footnote{This version is newer than the one we are using, but it's close enough.}.


\section{Applying the module}
This module is set up so that we may apply it to any of our server nodes and it will do the right thing. In the node definitions you have placed in \texttt{/etc/puppet/manifests/nodes.pp} we just need to add a line that says

\begin{verbatim}
    include ntp
\end{verbatim}

and the run the Puppet agent on each machine as we did earlier this week:

\begin{verbatim}
 sudo puppet agent --server=group00mgmt.foo.org.nz --no-daemonize --verbose --onetime
\end{verbatim}

Wow, that's a long command. It seems like it would be easier to alias it\footnote{Wink, wink, nudge, nudge.}.

\section{Epilogue}
With NTP services running on our VMs, their clocks will be more accurate \emph{eventually}. Abruptly changing the time on a server can cause Bad Things to happen, so NTP adjusts clocks gradually until they are sufficiently accurate. In our case it may take a couple of days.

\end{document}
