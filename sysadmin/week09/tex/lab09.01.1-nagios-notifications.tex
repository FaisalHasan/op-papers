\documentclass{article}   	% use "amsart" instead of "article" for AMSLaTeX format
\usepackage[margin=0.5in]{geometry}                		% See geometry.pdf to learn the layout options. There are lots.
\geometry{a4paper}                   		% ... or a4paper or a5paper or ...

\usepackage[parfill]{parskip}    		% Activate to begin paragraphs with an empty line rather than an indent
\usepackage{graphicx}				% Use pdf, png, jpg, or eps with pdflatex; use eps in DVI modeA
\usepackage{hyperref}
\usepackage{enumerate}								% TeX will automatically convert eps --> pdf in pdflatex		


\title{Lab 9.2:  Nagios Notifications\\ IN719 Systems Administration}
\date{}							% Activate to display a given date or no date

\begin{document}
\maketitle

\section*{Introduction}
Right now our Nagios servers are capable of sending us notifications via email when something happens.  This is good, but if something happens while we're not checking email, then we won't find out until later.  We would like to be able to push notifications to mobile devices so that we get alerts wherever we happen to be.  Probably the best approach is to set up a server that is capable of directly sending SMS messages, but in situations like ours that is not feasible.  Another approach is to use an SMS gateway service that accepts messages over the network via SMPP and forwards them via SMS.  The downside of this approach is that if your network goes down you also lose your ability to send a notification that the network is down, but sometimes that's a compromise we have to make.

If we're sending notifications over the network anyway, then there are some free messaging applications that we can use to send notifications to mobile devices.  One such application is \emph{Slack}.  It is available for most mobile platforms and it has an easily accessible API.  In this lab we'll set up our Nagios servers to send notifications using WhatsApp.

\section{Install Slack}
You will need to install the Slack mobile app on your device.  Log into the in719 Slack instance. If you haven't already, set up a channel for your team.

\section{Set up the Slack plugin for Nagios}

\begin{enumerate}
  \item Install the Perl modules \texttt{libwww-perl libcrypt-ssleay-perl}.
  \item  Download the plugin file from \url{https://raw.github.com/tinyspeck/services-examples/master/nagios.pl}.  Edit the file to set
  the values of \texttt{\$opt\_domain} and \texttt{\$opt\_token} as directed by the lecturer. 
  \item Copy the file to \texttt{/usr/lib/nagios/plugins} and set its permissions to 0755.
  \item Create a file \texttt{/etc/nagios-plugins/config/slack.cfg} with the following contents:
       \begin{verbatim}
define command {
      command_name notify-service-by-slack
      command_line /usr/lib/nagios/plugins/nagios.pl -field slack_channel=#alerts -field \
      HOSTALIAS="$HOSTNAME$" -field SERVICEDESC="$SERVICEDESC$"  \
      -field SERVICESTATE="$SERVICESTATE$"  -field SERVICEOUTPUT="$SERVICEOUTPUT$" \
      -field NOTIFICATIONTYPE="$NOTIFICATIONTYPE$"
}

define command {
      command_name notify-host-by-slack
      command_line /usr/lib/nagios/plugins/nagios.pl -field slack_channel=#ops -field \
      HOSTALIAS="$HOSTNAME$" -field HOSTSTATE="$HOSTSTATE$" \
      -field HOSTOUTPUT="$HOSTOUTPUT$" \
      -field NOTIFICATIONTYPE="$NOTIFICATIONTYPE$"
}
\end{verbatim}

  \item Now, for any contacts you define, you have the option of using for your host and service notification command the commands
  \texttt{notify-host-by-slack} and \texttt{notify-service-by-slack}.

\end{enumerate}
\end{document}
