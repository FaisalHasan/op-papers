\documentclass{article}
\usepackage{graphicx}
\usepackage{wrapfig}
%\usepackage{inconsolata}
\usepackage{enumerate}
\usepackage{hyperref}
\usepackage{verbatim}
\usepackage[parfill]{parskip}
\usepackage[margin = 2.5cm]{geometry}

\usepackage[T1]{fontenc}


\begin{document}

\title{Lab 03.2: Configuring NAT\\ IN720 Virtualisation}
\date{}
\maketitle

\section*{Introduction}
Right now our Xen systems use bridged networking for guest domains, which is the default behaviour. There are 
other other networking models available: routing and NAT. In this lab we will reconfigure our Xen hosts and guests 
to use NAT. This will give us a little more direct experience with how networking is set up in Xen and provide a chance 
to compare networking types. We've chosen NAT for this because it is an option that is functional in our larger 
networking environment.

Note that when we finish this lab we will want to restore bridged networking on our Xen hosts. To facilitate this, it 
would be a good idea to comment out items of configuration that we need to deactivate rather than deleting them.

\section{The plan}
We will modify our Xen networking configuration in several steps:

\begin{enumerate}
  \item Remove the bridged interface from Dom0;
  \item Enable routing on Dom0;
  \item Configure \texttt{xen} to use NAT;
  \item Modify our DomU \texttt{xl} configuration to use NAT;
  \item Modify our DomU guest configuration to use static network configuration (since DHCP will no longer be available).
\end{enumerate}

Since some of our network changes will interrupt our network connections, it may be necessary to do some of this work in the console rather then over ssh.

\section{Remove bridged networking}
Recall that we set up bridged networking on our Dom0 systems by installing the \texttt{bridge-utils} package and configuring a bridge interface in the file \texttt{/etc/network/interfaces}. To stop using bridging, we just need to remove 
the bridge interface and reconfigure the \texttt{em1} interface.

In the file \texttt{/etc/network/interfaces}, comment out the references to the xenbr0 interface. Find the line that says

\texttt{inface em1 inet manual} 

and change it to

\texttt{iface em1inet dhcp}

Don't worry about restarting the interfaces now, we will make additional changes first.

\section{Enable routing on Dom0} 
Enabling routing on Linux is extremely easy, Simply edit the file \texttt{/etc/sysctl.conf} and add (or just 
uncomment) the line

\texttt{net.ipv4.ip\_forward = 1}

We can make this change take immediately by entering the command

\texttt{sysctl -p /etc/sysctl.conf}

but we will be rebooting soon so it's not actually necessary.

\section{Configure \texttt{xen} to use NAT}
\texttt{xen} is configured to use bridging by default by referencing a script in \texttt{/etc/xen/scripts}. We just need to 
use a different script. Edit the file \texttt{/etc/xen/xend-config.sxp}. Look for a line that says 

\texttt{(network-script network-bridge)}

Comment it out and replace it with 

\texttt{(network-script network-nat)}

At this point we've made a number of changes, and a good way to make sure they all take effect is to use \texttt{xl} 
to shut down any guest VMs and reboot the host. When it comes back up, verify that Dom0 has network connectivity 
before proceeding.

\section{Enable NAT}
To enable NAT we just set a firewall rule on Dom0 with the command

\texttt{iptables -t nat -A POSTROUTING -o em1 -j MASQUERADE}

This setting will not persist if the host is rebooted. To do that we would need to add it to a script that is run at boot time, 
but for our purposes today that is not necessary.

\section{Modify DomU configuration}
We saw last time how the default \texttt{vif} settings work and cause our Dom0 guests to use bridged networking. 
Now we will modify the \texttt{vif} setting for one of our DomU guests to use NAT. In the guest config file (just choose one of yours), change 

\texttt{vif = ['']}

to 

\texttt{vif = ['ip=192.168.5.10,script=vif-nat']}

Note that we will use static ip addressing since we do not have a DHCP server. Start the guest VM after making this change.

\section{Modify the guest's networking configuration}
Although our Xen system is configured to use NAT, our guest operating system still is not. We need to configure 
the guest's \texttt{eth0} interface to use the new networking setup.

First, use \texttt{ifconfig} on Dom0 to inspect the settings of the backend interface for the guest. Note its ip address.

Next login to a console on the guest and edit its \texttt{/etc/network/interfaces} file, setting up eth0 as follows:

\begin{verbatim}

  iface eth0 inet static
    address 192.168.5.10
        broadcast 192.168.5.255
        netmask 255.255.255.0
        gateway <ip address of backend interface>
 \end{verbatim}
 
 No we just need to restart networking on the guest for the changes to take effect. If we get errors when doing this, it suffices to restart the guest. 
 
 Finally, verify that you have network connectivity on the guest by pinging an external host. Note that we do not have a 
 DNS resolver configured, so the guest won't find hosts by name.
 
 When you have done all this, you will want to restore bridged networking on you Xen system to prepare for upcoming 
 work.

\end{document}