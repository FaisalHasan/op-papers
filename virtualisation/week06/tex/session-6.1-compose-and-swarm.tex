% Beamer slide template prepared by Tom Clark <tom.clark@op.ac.nz>
% Otago Polytechnic
% Dec 2012

\documentclass[10pt]{beamer}
\usetheme{Dunedin}
\usepackage{graphicx}
\usepackage{fancyvrb}

\newcommand\codeHighlight[1]{\textcolor[rgb]{1,0,0}{\textbf{#1}}}

\title{Compose and Swarm}

\author[I720]{Virtualisation}
\institute[Otago Polytechnic]{
  Otago Polytechnic \\
  Dunedin, New Zealand \\
}
\date{}
\begin{document}

%----------- titlepage ----------------------------------------------%
\begin{frame}[plain]
  \titlepage
\end{frame}

\begin{frame}[fragile]
  \frametitle{To some extent, it just works}
  
  Make a \texttt{docker-compose.yml} file, then run
  
  \begin{verbatim}
  $ eval "$(docker-machine env --swarm <manager>)"
  $ docker-compose up
   \end{verbatim}
   
   There are some issues, however.
\end{frame}

\begin{frame}
  \frametitle{First issue, building}
   
   \begin{itemize}
     \item We saw that you can specify that Docker build an image in a Compose file.
     \item But we also saw that Swarm needs images to be accessible in a registry.
     \item So, building in a compose file used for Swarm deployment is right out.
     \item Really, when you're deploying to a production setting, you shouldn't be building on the fly anyway.
   \end{itemize}
\end{frame}

\begin{frame}
  \frametitle{Next issue, volumes}
   
   \begin{itemize}
     \item We use volumes to let containers work directly with the host file system.
     \item But we don't always know what host our containers will run upon.
     \item If our container creates volumes, use named ones.
     \item If our containers read from volumes, have other containers that populate them, then use a \texttt{volumes\_from} directive.
   \end{itemize}
\end{frame}

\begin{frame}
  \frametitle{Issue three: dependencies}
   
   \begin{itemize}
     \item We can create dependencies between containers
              \begin{itemize}
                \item Explicitly: \texttt{depends\_on}
                \item Implicitly: \texttt{volumes\_from}
              \end{itemize}
         
     \item In either case, Docker interprets this to mean that a container must be deployed on the same node as its dependencies.
     \item In simple cases this works fine.
     \item We need to be aware of this when scaling services. When scaling a service, we may need to scale its dependencies.
   \end{itemize}
\end{frame}


\begin{frame}[fragile]
    \frametitle{The tricky case: multiple dependencies}
    
    \begin{verbatim}
version: '2'
services:
    svc_a:  
        image: image_a
        depends_on:
            - svc_b
            - svc_c
    svc_b:  
        image: image_b
    svc_c:  
        image: image_c
    \end{verbatim}
\end{frame}

\begin{frame}
  \frametitle{Final issue: networks}
   
   \begin{itemize}
     \item We've seen that Docker can set up internal networks that allow containers to communicate and find each other by name.   
     \item What happens when the containers are on seperate hosts?
     \item We can get the same result by creating \emph{overlay networks}.
   \end{itemize}
\end{frame}

\begin{frame}[fragile]
    \frametitle{In compose}
     \begin{verbatim}
networks:
     foo:
         driver: overlay
     \end{verbatim}
 \end{frame}
 
 \begin{frame}
  \frametitle{Final caveats}
   
   \begin{itemize}
     \item This stuff is highly version-dependent.
     \item Documentation is sometimes lacking and what's there can be confusing.
     \item Stack Overflow does not care about you.
   \end{itemize}
\end{frame}    
 

\end{document}
