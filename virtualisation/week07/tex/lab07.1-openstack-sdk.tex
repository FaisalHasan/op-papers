\documentclass{article}
\usepackage{graphicx}
\usepackage{wrapfig}
%\usepackage{inconsolata}
\usepackage{enumerate}
\usepackage{hyperref}
\usepackage{verbatim}
\usepackage[parfill]{parskip}
\usepackage[margin = 2.5cm]{geometry}

\usepackage[T1]{fontenc}


\begin{document}

\title{Lab 7.1: Using the OpenStack SDK \\ IN720 Virtualisation}
\date{}
\maketitle

\section*{Introduction}
In this lab we will use the OpenStack SDK to access services in our and launch a VM instances. Our process will be to

\begin{enumerate}
  \item Get a connection 
  \item Identify our image
  \item Identify our flavour
  \item Find a suitable network
  \item Identify a keypair
  \item Launch an instance
  \item Associate a floating IP address with the instance.
\end{enumerate}

\section{Preliminary steps}
Before you start coding, create a subdirectory of your home directory to hold your work, unless you have already done so. In that directory, make sure there is a \texttt{clouds.yml} file that we set up the last lab. Activate your Python virtualenv and make sure you can create a working connection object like we saw last time. In the code below, we will assume that you have created an OpenStack connection object and associated it with the variable \texttt{conn}.

\section{Identify resources}
Before we can launch our instances, we need to find the needed OpenStack resources. Start by creating variables in your code with the resource names.

\begin{verbatim}
  IMAGE = 'ubuntu-16.04-x86_64'
  FLAVOUR = 'c1.c1r1'7
  NETWORK = 'private-net'
  KEYPAIR = '<your keypair name>'
\end{verbatim}

N.B.: You should have a keypair from previous labs. If not, just create a new pair.


\newpage

Next, we will use our connection object to get the resources with these names from our cloud.

\begin{verbatim}
    image = conn.compute.find_image(IMAGE)
    flavour = conn.compute.find_flavor(FLAVOUR)
    network = conn.network.find_network(NETWORK)
    keypair  = conn.compute.find_keypair(KEYPAIR)
\end{verbatim}

Now we have objects that represent the specific resources we need to launch an instance. Note that if you don't have a keypair ready to use on the cloud, then you can use the SDK to create one. These calls generally raise \texttt{ResourceNotFound} exceptions if the requested resource is not found.

\section{Launch an instance}
With our resources in hand, launching an instance is straightforward:

\begin{verbatim}
    SERVER = '<your username>-server'
    server = conn.compute.create_server(
        name=SERVER, image_id=image.id, flavor_id=flavour.id,
        networks=[{"uuid": network.id}], key_name=keypair.name) 
\end{verbatim}

Notice that the \texttt{networks} argument is specified slightly differently than the others. This is because we can connect a server to multiple networks, so we pass in a list of them.

The call is completed \emph{asynchronously}, meaning it may take some time before the instance is actually launched. We cannot proceed to the next step until this is done, so we will wait:

\begin{verbatim}
    server = conn.compute.wait_for_server(server)
\end{verbatim}


\section{Associate a floating IP address}
Floating IP addresses are available on our public network, so we will access the network and get an ip address from it:@

\begin{verbatim}
    public_net = conn.network.find_network('public-net')
    floating_ip = conn.network.create_ip(floating_network_id=public_net.id) 
\end{verbatim}

Now we can associate the address with the server

\begin{verbatim}
 conn.compute.add_floating_ip_to_server(server, floating_ip.floating_ip_address)
\end{verbatim}

\section{Cleaning up}
Don't forget to remove your instance from the cloud. Seeing how to do this with Python is left as an exercise. See

\url{https://docs.openstack.org/openstacksdk/latest/user/index.html}

for more information.

\end{document}