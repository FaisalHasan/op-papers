\documentclass{article}
\usepackage{graphicx}
\usepackage{wrapfig}
%\usepackage{inconsolata}
\usepackage{enumerate}
\usepackage{hyperref}
\usepackage{verbatim}
\usepackage[parfill]{parskip}
\usepackage[margin = 2.5cm]{geometry}

\usepackage[T1]{fontenc}


\begin{document}

\title{Assignment 3: Xen \\ IN720 Virtualisation}
\date{}
\maketitle

\section*{Introduction}
In this assignment you will write a detailed HOWTO document describing the process of creating a parvirtualised
guest image and then creating additional guest cloned from this image. Your target audience for this document is
a typical student approaching the end of the first year in the BIT.

Your document should include a complete step-by-step procedure for carrying out the tasks along with some disscussion
of \emph{why} the various steps are necessary and what the effects of them are. You should also call attention to 
possible problems that can arise and how to avoid or fix them.

This assignment is worth 25\% of your mark in this paper.

\section{Specifications}
.
For your document, you may assume that the person using it is working with a single server running Xen 4.4 with an 
Ubuntu Dom0 operating system, and that the hosts disk drive(s) are set up with LVM. Also assume that the Xen host
is configured with bridge networking and an external DHCP server. (Basically, the same system that you have been using for this paper.

Your document should explain how to do the following things:

\begin{enumerate}
	\item Set up a volume for a guest image.
	\item Obtain files necessary for running a Linus installer and place them on the Xen host.
	\item Boot a guest running the installer and carry out the installation.
	\item Perform any post installation finalisation of the guest image.
	\item Prepare volumes for cloned guests.
	\item Write Xen configuration files for the cloned guests.
	\item Create the guest domains.
	\item Verify basic functionality of the running guests.
\end{enumerate}

Again, be sure your document explains both what to do and why it needs to be done (Or not done - if
doing nothing and taking the default behaviour, be clear about that and explain what that default is.).

\section{Submission}
You assignment will be evaluated on its clarity, correctness, and completeness. Take advantage of formatting to be
sure that it is clear to the reader what thing are commands to be entered in a terminal, associated output, and your 
explanation. You may find it useful to include screen shots.

This assignment is due on Monday, 19 November at noon.  Email the document in PDF format to the lecturer at or before this time.

\end{document}